% !TEX root =  main.tex

Initial evidence has been established that machine learning techniques are capable of identifying (non- linear) structures in financial market data, see Huck (2009, 2010),  
Takeuchi and Lee (2013) , Moritz and Zimmermann (2014) , Dixon, Klabjan, and Bang (2015) , and further references in Atsalakis and Valavanis (2009) as well as Sermpinis, Theofilatos, Karathana- sopoulos, Georgopoulos, and Dunis (2013). The relevant work on deep learning applied to finance has introduced the convolutional neural networks, deep belief networks. For example, Ding et al.~\cite{ding2015deep} combine the neural tensor network and the deep convolutional neural network to predict the short-term and long-term influences of events on stock price movements. Other than that, certain works use deep belief networks in financial market prediction, for example, Yoshihara et al. ~\cite{yoshihara2014predicting}, Shen et al. ~\cite{shen2015forecasting} and Kuremoto et al. ~\cite{kuremoto2014time}.
 
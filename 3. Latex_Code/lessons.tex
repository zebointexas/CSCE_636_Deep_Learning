During the course of this project, we attempted to extend \scheme\ with several ideas. 
Not all worked out. The failed attempts along with some explanation 
%of the failures 
are as follows: 

%\begin{itemize}
%\begin{enumerate}%[wide, labelwidth=!, labelindent=0pt]    
    %\item 
    {\em 1. Treating $f^{CF}$ and $f^{LCS}$ as regression instead of classification 
    problem:} When we considered $f^{CF}$ and $f^{LCS}$ as regression problems, the neural
    network produced higher prediction error. 
    Therefore, the performance of \scheme\ degraded.

    %\item 
    {\em 2. Neural network to predict the order of genes:} The goal of the fitness 
    score is to provide an order among genes so that the Roulette Wheel algorithm works. So, 
    instead of predicting the actual scores, we tried to predict the order among genes. This 
    idea did not work out because ordering accuracy was worse than that of 
    fitness score based approach.
    
    %\item 
    {\em 3. Multiple predictors:} We experimented with one neural network 
    to predict whether 
    a gene has the fitness score of 0 or not. In case, the fitness score was predicted to be non-zero, we used a different neural network to predict the actual non-zero value. This idea came from the intuition that since many genes have a fitnesss score of 0 (at least for initial generations), we can do a better job predicting those if we use a separate predictor for that purpose. Unfortunately, this ideas also did not work out because the first predictor failed to identify some of good genes that appeared during initial generations.
    
    %\item 
    {\em 4. NS to improve gene quality:} NS can also be used to find better quality genes which can replace the top quality genes that \scheme\ started with (during NS).
    However, since the neural network fitness function is not any more accurate on neighborhood genes than it is on genes generated by the evolutionary algorithm, \scheme\ failed to identify actual better quality genes from the neighborhood in many cases. %That is why, this idea did not benefit \scheme\.     

    %\item 
    {\em 5. Knowing the target program length:} Unsurprisingly, we have found that the \scheme\ finds programs most quickly when we have a priori knowledge of the target program length and maintain all genes at that length.  It is possible to generate the initial genes with some length distribution %centered around some mean 
    and then allow crossover and mutation to change gene length but that increases the time to solution significantly if the target program length is much larger or smaller than the mean initial gene length.  In the future, we plan to explore using a neural network to predict program length to remove the need for a priori knowledge.
    
        %\item {\em }
%\end{itemize}
%\end{enumerate}
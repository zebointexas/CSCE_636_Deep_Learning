\begin
{abstract}
The application of neural network (NN) methods in finance has received a great deal of attention from both industry and academia ~\cite{bao2017deep}. This study presents an integrated analysis which combines stacked autoencoders, long short term memory (LSTM) and feature analysis with embedded NN. LSTM networks are originally designed for sequential learning but are less commonly used in financial time series predictions; yet, they are inherently suitable for this domain~\cite{fischer2018deep}. We deployed LSTM networks for predicting out-of-sample directional movements for the constituent stocks of the S\&P500 from 2009 until 2018. According to the result, outperformance relative to the general market is very clear from 1992 to 2009, but as of 2010, excess returns seem to have been arbitraged away with LSTM profitability fluctuating around zero after transaction costs. We further unveil sources of profitability, thereby shedding light into the black box of artificial neural networks. The neural network has three stages. {\em First}, the stock price time series is decomposed by WT to eliminate noise. {\em Second}, SAEs is applied to generate deep high-level features for predicting the stock price. {\em Third}, high-level denoised features are then fed into LSTM to forecast the next day's closing price. S\&P500 index and their corresponding index futures are chosen to examine the performance of the proposed model. 
%Finally we compared the performance with other techniques such as SVM, SVR and XGBoost. 
\end
{abstract}